%\documentclass[manuscript]{aastex}
\documentclass[iop,numberedappendix]{emulateapj}
\usepackage{natbib}
\usepackage{amsmath}
\usepackage{booktabs}
\usepackage{color}
\usepackage{graphicx, subfigure}
\citestyle{aa}
\usepackage{CJKutf8}
\interfootnotelinepenalty=10000
\newcommand{\Munan}[1]{{\color{red}#1}}
\newcommand{\di}{\mathrm{d}} 
\newcommand{\e}{\mathrm{e}} 
\newcommand{\mr}{\mathrm} 
\newcommand{\Ht}{\mathrm{H_2}} 
\newcommand{\Ho}{\mathrm{H}} 
\newcommand{\HI}{\mathrm{HI}} 
\newcommand{\Hplus}{\mathrm{H^+}}
\newcommand{\He}{\mathrm{He}} 
\newcommand{\Heplus}{\mathrm{He^+}}
\newcommand{\HCOplus}{\mathrm{HCO^+}}
\newcommand{\CO}{\mathrm{CO}} 
\newcommand{\CI}{\mathrm{CI}}
\newcommand{\OI}{\mathrm{OI}}
\newcommand{\Cplus}{\mathrm{C^+}} 
\newcommand{\CHx}{\mathrm{CH_x}} 
\newcommand{\CH}{\mathrm{CH}} 
\newcommand{\OHx}{\mathrm{OH_x}} 
\newcommand{\Oplus}{\mathrm{O^+}} 
\newcommand{\Si}{\mathrm{Si}} 
\newcommand{\Siplus}{\mathrm{Si^+}} 


\defcitealias{GO2015}{GO15}

\begin{document}
\title{Improved heating and cooling in TIGRESS simulations}
\author{Munan Gong (\begin{CJK*}{UTF8}{gbsn}龚慕南\end{CJK*})\altaffilmark{1}}
\altaffiltext{1}{Max-Planck Institute for Extraterrestrial Physics,
Garching by Munich, 85748, Germany; 
munan@mpe.mpg.de}

\begin{abstract}
    Heating and cooling functions for neutral and molecular ISM in 
    TIGRESS simulations \citep{KO2017}
    based on equilibrium chemistry in \citet{GOW2016}.
\end{abstract}

\section{Introduction}
The aim here is to develop physically motivated accurate 
heating and cooling functions for the neutral and molecular ISM 
($T<10^4~\mr{K}$ gas) that can better capture the gas thermo-dynamics than the
simple heating and cooling functions from \citet{KI2002}. 
For $T>10^4~\mr{K}$, the original implementation of
collisional ionization equilibrium (CIE) cooling from \citet{SD1993} 
should be used. 

The heating and cooling functions in this work 
take the hydro and radiation variables in the TIGRESS simulation and
give the heating and cooling rates. The method here gives  
results similar to the equilibrium chemistry, but does not require solving
the full chemistry ODEs. The abundances of
different species are calculated (semi-)analytically from equilibrium conditions.
The heating and cooling rates can be subsequently calculated from analytic expressions
or interpolation tables (in the case of $\CO$ rotational lines).

Section \ref{section:para} gives a summary of the input and output parameters.
Section \ref{section:chem} explains the calculation of the
abundances of chemical species. Section \ref{section:heating_cooling} 
describes the heating and cooling processes included. Section
\ref{section:code_tests} presents some tests for the heating and cooling
functions. Finally, Section \ref{section:notes} lists some notes for the
implementation in the Athena-TIGRESS code.

\section{Input and output parameters}\label{section:para}
Table \ref{table:input} list all the input parameters. The outputs are the
heating rate $\Gamma$ and cooling rate $\Lambda$:
\begin{equation}\label{eq:heating}
    \Gamma(x_i, T, Z, \xi_H, G_\mr{PE}, G_\Ht) = \Gamma_\mr{PE} + \Gamma_\mr{CR} +
    \Gamma_\mr{H_2 pump},
\end{equation}
and
\begin{align}\label{eq:cooling}
    \begin{split}
    &\Lambda(x_i, n_\Ho, T, \langle |\di v/\di r| \rangle, Z, G_\mr{PE}) \\
    &= \Lambda_\mr{Ly\alpha}+\Lambda_\OI + \Lambda_\Cplus +
    \Lambda_\CI + \Lambda_\CO + \Lambda_\mr{rec}.
    \end{split}
\end{align}
The relevant chemical abundances $x_i$ are listed in Table \ref{table:chem},
and can be calculated from the input parameters.

The details about these heating and cooling processes are explained in the
following sections.

\begin{table*}[htbp]
    \centering
    \caption{Input parameters}
    \label{table:input}
    \begin{tabular}{l l l l}
        \tableline
        \tableline
        Symbol & Meaning & Code Units & Note\\ 
        \tableline
        \multicolumn{4}{l}{Hydro parameters:}\\
        $n_\Ho$ & number density of hydrogen atoms & $\mr{cm^{-3}}$ 
        &$\rho=1.4271 n_\Ho m_\Ho$
        \\
        $T$ & temperature &$\mr{K}$ 
        &$T=\mu m_\Ho P/ (\rho k_b)$, $\mu$ is the molecular weight\tablenotemark{a} 
        \\
        $\langle |\di v/\di r| \rangle$ 
        & the mean (absolute) velocity gradient\tablenotemark{b} &$\mr{s^{-1}}$
        & for LVG approximation in $\CO$ cooling 
        \\
        \multicolumn{4}{l}{Radiation field strengths\tablenotemark{c}:}\\
        $G_\mr{PE}$ &photo-electric heating &$G_0$ &
        $\gamma_\mr{PE} = 1.87$
        \\
        $G_\CI$ &radiation field for $\CI$ to $\Cplus$ photo-ionization 
        &$G_0$ &$\gamma_\CI = 3.76$
        \\
        $G_\CO$ &$\CO$ photo-dissociation (only dust
        shielding)&$G_0$ & $\gamma_\CO = 3.88$. \Munan{Use the same $\gamma$
        for $\CI$ and $\CO$?}
        \\
        $G_\Ht$\tablenotemark{d} 
        &$\Ht$ photo-dissociation (dust- and self-
        shielding) &$G_0$ &dust shielding: $\gamma_\Ht=4.18$,\\
        & & &self-shielding from \citet{DB1996}.
        \\
        \multicolumn{4}{l}{Other parameters:}\\
        $Z$ & metallicity &$Z_\odot$
        &the same metallicity for gas and dust $Z=Z_d=Z_g$
        \\
        $\xi_\Ho$ & primary cosmic-ray ionization rate per $\Ho$ atom
        & $\mr{s^{-1}H^{-1}}$
        & scales with star formation rate and surface density\tablenotemark{e}
        \\

        \tableline
    \end{tabular}
    \tablenotetext{1}{$\mu = [m_\mathrm{He, tot} x_\mathrm{He, tot} +
    m_\Ht x_\Ht + m_\Ho (1-x_\Ht) ]/ [m_\Ho (x_\mathrm{He,tot} + x_\Ht +
    (1-2x_\Ht) + x_\e)] = (m_\mathrm{He, tot} x_\mathrm{He, tot} + m_\Ho) /
    [m_\Ho (x_\mathrm{He,tot} + 1 - x_\Ht + x_\e)] = 
    1.4271 / (x_\mathrm{He,tot} + 1 - x_\Ht + x_\e)$. $T=P/[n_\Ho k_b
    (x_\mathrm{He,tot} + 1 - x_\Ht + x_\e)]$.}
    \tablenotetext{2}{Averaged across the six faces of each grid cell in the
    simulation.}
    \tablenotetext{3}{For one-sided slab and only consider dust shielding,
    $G_i = G_0 \exp(-\gamma_i A_V) = G_0 \exp(-\sigma_i N_\Ho)$,
    where $G_0 = 2.7 \times 10^{-3} \mr{erg~cm^{-2} s^{-1}}$
    is the interstellar radiation field in \citet{Draine1978}
    (\Munan{The spectrum from star clusters might be different?}), $A_V=N_\Ho/1.87\times
    10^{-21}\mr{cm^{-2}}$, and the cross-section $\sigma_i = (\gamma_i/1.87) \times
    10^{21}~\mr{cm^{-2}}$. }
    \tablenotetext{4}{We only need to calculate $G_\Ht$ if we want to include
    heating by UV-pumping of $\Ht$ (in high radiation field and low metallcitiy
    gas) or we want to calculate $\Ht$ abundances in cloud edges including FUV
    photo dissociation (not necessary in most cases, since $\Ht$ is considered
    to be affected by non-equilibrium chemistry).}
    \tablenotetext{5}{For the solar neighborhood,
    $\xi_\Ho \approx 2\times 10^{-16}~\mr{s^{-1} H^{-1}}$. 
    $\xi_{-16} = 0.472
    \frac{\Sigma_\mathrm{SFR,-3}}{\Sigma_\mathrm{gas}/50M_\odot\mathrm{pc^{-2}}
    + 1}$, $\xi_{-16} = \xi_\Ho / 10^{-16}\mr{s^{-1} H^{-1}}$, 
    $\Sigma_\mathrm{SFR,-3} =
    \Sigma_\mathrm{SFR}/10^{-3}M_\odot\mathrm{kpc^{-2}Myr^{-1}}$.}
\end{table*}

\section{Chemical Abundances}\label{section:chem}
In order to calculate the heating and cooling rates in Equations
(\ref{eq:heating}) and (\ref{eq:cooling}), we need to know the chemical
abundances of the species listed in Table \ref{table:chem}. We explain the
calculation of the abundances of these species below. 
\begin{table*}[htbp]
    \centering
    \caption{Chemical species}
    \label{table:chem}
    \begin{tabular}{l l l}
        \tableline
        \tableline
        Species & Abundance calculation & Dependence\\ 
        \tableline
        $\Ht$ & analytic, see Section \ref{section:H2}
        & $n_\Ho$, $T$, $Z_d$, $\xi_\Ho$ (and $G_\Ht$)
        \\
        $\mr{e^-}$ & iterative, assumes $x_\e=x_\Hplus + x_\Cplus$, 
        see Section \ref{section:e} 
        & $x_\Ht$, $n_\Ho$, $T$, $Z_d$, $Z_g$, $\xi_\Ho$, $G_\mr{PE}$, $G_\CI$
        \\
        $\Cplus$ &analytic, see Section \ref{section:e} 
        & $x_\e$, $x_\Ht$, $n_\Ho$, $T$, $Z_d$, $Z_g$, 
        $\xi_\Ho$, $G_\mr{PE}$, $G_\CI$
        \\
        $\Hplus$ & analytic, see Section \ref{section:e} 
        & $x_\e$, $x_\Ht$, $x_\Cplus$, $n_\Ho$, $T$, $Z_d$,
        $\xi_\Ho$, $G_\mr{PE}$
        \\
        $\CO$ &analytic, see Section \ref{section:CO} 
        & $x_\Ht$, $x_\Cplus$, $n_\Ho$, $Z_d$, $Z_g$, 
        $\xi_\Ho$, $G_\CO$
        \\
        $\HI$ &elemental conservation & $x_\HI = 1 - 2x_\Ht - x_\Hplus$ 
        \\
        $\CI$ &elemental conservation & $x_\CI = x_\mr{C,tot} - x_\Cplus -
        x_\CO$ 
        \\
        $\OI$ &elemental conservation & $x_\OI = x_\mr{O,tot} - x_\CO$ 
        \\
        \tableline
    \end{tabular}
\end{table*}

\subsection{$\Ht$ Abundance\label{section:H2}}
The $\Ht$ abundance can be obtained from Equation (18) in \citet{GOK2018}:
\begin{equation}\label{eq:GOK2018_eq18}
    f_\Ht n_\Ho k_\mr{gr} = 1.65 f_\Ht k_\mr{CR}.
\end{equation}
The left hand side is the rate of $\Ht$ formation on dust grains. On the right
hand side, $f_\Ht k_\mr{CR}$ is the destruction of $\Ht$ by cosmic rays, and
the $1.65$ factor comes from additional channels of $\Ht$ destruction by
$\mr{H_2^+}$ and $\Ht$ formation by $\mr{H_3^+}$. If we take the
photo-dissociation of $\Ht$ by FUV radiation into account, which can be
important at the edge of the cloud especially when the cosmic ray ionization
rate is low, then Equation (\ref{eq:GOK2018_eq18}) becomes:
\begin{equation}\label{eq:H2}
    f_\Ht n_\Ho k_\mr{gr} = 1.65 f_\Ht k_\mr{CR} + f_\Ht k_\gamma,
\end{equation}
where $k_\gamma=5.7\times 10^{-11}G_\Ht~\mr{s^{-1}}$ is the photo-dissociation
rate of $\Ht$. Using $x_\HI = 1-2x_\Ht$ and
$k_\mr{CR}=2\xi_\Ho(2.3x_\Ht+1.5x_\HI)$, Equation (\ref{eq:H2}) can be written
as a quadratic equation for $x_\Ht$:
\begin{align}
    &a x_\Ht^2 + bx_\Ht + c = 0\\
    &a = 1.155\\
    &b=- (2.475 + 2R + \frac{k_\gamma}{2\xi_\Ho}),
    ~R=\frac{k_\mr{gr}n_\Ho}{2\xi_\Ho}\\
    &c=R,
\end{align}
and the $\Ht$ abundance $x_\Ht=(-b - \sqrt{b^2 - 4ac} )/(2a)$. If $k_\gamma=0$,
this recovers the result in \citet{GOK2018} without photo-dissociation.

\subsection{$\Cplus$, $\Hplus$ and $\mr{e^{-}}$ Abundances\label{section:e}}
The $\Cplus$ equilibrium abundance is calculated from the balancing of the
$\Cplus$ creation by cosmic-ray and FUV ionisation
\begin{align*}
    \mr{cr + C }&\mr{\rightarrow C^+ + e}\\
    \mr{\gamma + C }&\mr{\rightarrow C^+ + e}\\
\end{align*}
and $\Cplus$ destruction by recombination (gas phase and on the grain surface)
and reaction with $\Ht$
\begin{align*}
    \mr{C^+ + e }&\mr{\rightarrow\ C}\\
    \mr{C^+ + e + gr }&\mr{\rightarrow C + gr}\\
    \mr{C^+ + H_2 }&\mr{\rightarrow CH_2}
\end{align*}

On of the main creation and destruction channels for $\Hplus$ is 
$\mr{O^+ + H \rightarrow H^+ + O}$ and $\mr{H^+ + O \rightarrow O^+ + H}$. 
These two reactions are also the dominant channels for $\Oplus$ destruction and
creation. The equilibrium of $\Oplus$ requires this two reactions to balance
each other. Therefore, the $\Hplus$ creation and destruction from these two
reactions cancel out, and we can obtain the $\Hplus$ abundances from the
remaining important creation and destruction channels. We consider the $\Hplus$
creation by cosmic ray ionization
\begin{equation*}
    \mr{cr + H\rightarrow H^+ + e}
\end{equation*}
and destruction by recombination in gas phase and grain surface
\begin{align*}
    \mr{H^+ + e}&\mr{\rightarrow H}\\
    \mr{H^+ + e + gr}&\mr{\rightarrow H + gr}
\end{align*}

We assume most electrons comes from $\Hplus$ and $\Cplus$, which is true except
for very shielded regions where the electron abundance is already very low
\begin{equation*}
    x_\e = x_\Hplus + x_\Cplus.
\end{equation*}
The neutral $\Ho$ is calculated assuming all hydrogen is in the from of $\Ho$,
$\Ht$ or $\Hplus$
\begin{equation*}
    x_\HI + x_\Ht + x_\Hplus = 1.
\end{equation*}
This is a good assumption, since the abundances of other species that contains
hydrogen, $\mr{H_3^+}$, $\mr{H_2^+}$, $\CHx$, $\OHx$ and $\mr{HCO^+}$ is very
low comparing to $\Ho$, $\Ht$ and $\Hplus$.

With all the conditions above, the electron abundance can be solved
iteratively. We use a iteration scheme similar to the Dekker's method
(a combination of the secant method and bisection method) in root finding, 
and found that the electron abundance
is converged within about 10 iterations. After the electron abundance is
obtained, $\Cplus$ and $\Hplus$ abundance can then be calculated from the
equilibrium conditions.

\subsection{$\CO$ Abundance\label{section:CO}}
The $\CO$ abundance is calculated making use of the fitting function Equation
(25) in \citet{GOW2016}:
\begin{equation}\label{eq:CO_fit}
    \frac{n_\mr{crit, CO}}{\mr{cm^{-3}}} 
    = \left(4\times10^3 Z \xi_{\Ho,16}^{-2}\right)^{G_\CO^{1/3}}
    \left( \frac{50  \xi_{\Ho,16}}{Z^{1.4}}\right),
\end{equation}
where $n_\mr{crit, CO}$ is the critical value above which 
$x_\CO/x_\mr{C, tot} > 0.5$, $x_\mr{C, tot}$ the total carbon abundance, and
$\xi_{\Ho,16}=\xi_\Ho/(10^{-16}\mr{s^{-1}H^{-1}})$.
We set the $\CO$ abundance 
\begin{equation*}
    \frac{x_\CO}{x_\mr{C, tot}} =
    \begin{cases}
         1, & n_\Ho \geq 2 n_\mr{crit, CO}\\
         \frac{n_\Ho}{2 n_\mr{crit, CO}}, & n_\Ho < 2 n_\mr{crit, CO}
    \end{cases}
\end{equation*}
In addition, we put upper limits on the $\CO$ abundance, with 
$x_\CO \leq x_\mr{C, tot} - x_\Cplus$ (conservation of the carbon atoms)
and $x_\CO/x_\mr{C, tot} \leq 2 x_\Ht$
(because $\Ht$ is a prerequisite for $\CO$ formation).

\section{Heating and Cooling}\label{section:heating_cooling}
The heating and cooling processes included are listed in Table
\ref{table:thermo}. For the details, please see \citet{GOW2016}.
\begin{table*}[htbp]
    \centering
    \caption{List of Heating and Cooling Processes}
    \label{table:thermo}
    \begin{tabular}{l  l}
        \tableline
        \tableline
        Process &Dependence\\ 
        \tableline
        \multicolumn{2}{l}{Heating:}\\
        Cosmic-ray ionization of $\Ho$, $\Ht$ and $\He$ 
        &$x_\e$, $x_\HI$, $x_\Ht$, $n_\Ho$, $\xi_\Ho$\\
        Photoelectric effect on dust grains 
        &$x_\e$, $n_\Ho$, $T$, $Z_d$, $G_\mr{PE}$ \\
        UV pumping of $\Ht$\tablenotemark{a}
        &$x_\HI$, $x_\Ht$, $n_\Ho$, $T$, $G_\Ht$\\

        \multicolumn{2}{l}{Cooling:}\\
        Ly$\alpha$ line & $x_\e$, $x_\HI$, $n_\Ho$, $T$\\
        $\mr{O}$ fine structure line 
        &$x_\e$, $x_\OI$, $x_\HI$, $x_\Ht$, $n_\Ho$, $T$\\
        $\mr{C^+}$ fine structure line 
        &$x_\e$, $x_\Cplus$, $x_\HI$, $x_\Ht$, $n_\Ho$, $T$\\
        $\mr{C}$ fine structure line 
        &$x_\e$, $x_\CI$, $x_\HI$, $x_\Ht$, $n_\Ho$, $T$\\
        $\CO$ rotational lines 
        &$x_\e$, $x_\CO$, $x_\HI$, $x_\Ht$, $n_\Ho$, $T$, 
        $\langle |\di v/\di r| \rangle$\\
        Recombination of $\mr{e}$ on PAHs 
        &$x_\e$, $n_\Ho$, $T$, $Z_d$, $G_\mr{PE}$\\
        \tableline
    \end{tabular}
    \tablenotetext{1}{Important at low metalicities. \Munan{reference?}}
\end{table*}

\section{Code Tests}\label{section:code_tests}
The benchmark for our model is the heating and cooling
rates from equilibrium chemistry calculations in \citet{GOW2016}. We also
compare our results with the cooling function from \citet{KI2002}.
The comparisons below 
are made with unshielded (except for $\Ht$ because of its very efficient
self-shielding) gas in solar neighbourhood and low metallicity conditions:
$Z=0.1, 1$, 
$\xi_\Ho = 2\times 10^{-16}~\mr{s^{-1} H^{-1}}$, $G_\mr{PE}=G_\CI=G_\CO=1$ 
(in \citet{Draine1978} units), $G_\Ht=0$, and
$\langle |\di v/\di r| \rangle = 9\times 10^{-14}~\mr{s^{-1}}$.

The comparisons are shown in Figure \ref{fig:code_tests_sn}
\ref{fig:code_tests_Z0p1}. The heating and
cooling rates from this work agree with that from the equilibrium chemistry in
\citet{GOW2016} within a factor of $\sim 2$. Compared to the results from the
equilibrium chemistry model, the heating rates from \citet{KI2002} 
are much lower, but the cooling rates are in good agreement. There is a larger
difference when the temperature is not in equilibrium, especially for the
cooling rates (gray lines in Figure
\ref{fig:code_tests_sn} and \ref{fig:code_tests_Z0p1}).

\begin{figure*}[htbp]
     \begin{center}
%
        \subfigure[Heating]{%
            \includegraphics[width=0.49\textwidth]{../fig/heating.pdf}
        }%
        \subfigure[Cooling]{%
           \includegraphics[width=0.49\textwidth]{../fig/cooling.pdf}
        } 

    \end{center}
    \caption{Comparisons of heating and cooling rates for $Z=1$. The black solid lines
    are the heating and cooling functions from \citet{KI2002} used in
    \citet{KO2017}. The blue solid lines are the results from equilibrium
    chemistry and temperature calculations by \citet{GOW2016} at densities 
    $n_\Ho=0.1-1000~\mr{cm^{-3}}$. The equilibrium temperature from
    \citet{GOW2016} at different densities is shown in Figure \ref{fig:nH_T}.
    The red solid lines show the heating and cooling rates
    from this work with the gas at these densities and equilibrium temperatures.
    The gray lines show the results from this work with fixed densities 
    $n_\Ho=0.1~\mr{cm^{-3}}$ (dotted), $n_\Ho=1~\mr{cm^{-3}}$ (dash-dotted), 
    and $n_\Ho=100~\mr{cm^{-3}}$ (dashed).
        \label{fig:code_tests_sn}
    }
\end{figure*}

\begin{figure*}[htbp]
     \begin{center}
%
        \subfigure[Heating]{%
            \includegraphics[width=0.49\textwidth]{../fig/heating_Z0p1.pdf}
        }%
        \subfigure[Cooling]{%
           \includegraphics[width=0.49\textwidth]{../fig/cooling_Z0p1.pdf}
        } 

    \end{center}
    \caption{caption.
        \label{fig:code_tests_Z0p1}
    }
\end{figure*}

\begin{figure}[htbp]
\centering
\includegraphics[width=\linewidth]{../fig/equilibrium_temperature.pdf}
\caption{caption}
\label{fig:nH_T}
\end{figure}



\section{Notes for Implementation in TIGRESS}\label{section:notes}
\begin{enumerate}
    \item The input parameters must be in CGS units listed in Table
        \ref{table:input}.
    \item The output heating and cooling rates are in units of
        $\mr{erg~s^{-1}~cm^{3}}$.
    \item Because the heating and cooling rates has a step of dividing by
        $n_\Ho$, one needs to make sure that $n_\Ho > 0$ (use density floor). 
    \item If we want to ignore the FUV dissociation of $\Ht$ ($\Ht$ abundance
        from only cosmic ray destruction, no heating from $\Ht$ UV pumping),
        one just need to set $G_\Ht=0$ in the code.
\end{enumerate}


\bibliographystyle{apj}
\bibliography{apj-jour,thesis}
\end{document}

\citet[][hereafter \citetalias{GO2015}]{GO2015}

\begin{figure}[htbp]
\centering
\includegraphics[width=\linewidth]{somefig.pdf}
\caption{caption}
\label{fig:somefig}
\end{figure}

\begin{table*}[htbp]
    \caption{caption}
    \label{table:some table}
    \begin{tabular}{l l l l l}
        \tableline
        \tableline
        No. &Reaction &Rate coefficient\tablenotemark{a} &Notes
        &Reference\\ 
        \tableline
        \multicolumn{5}{l}{Grain-assisted reactions:}\\
        1 &$\mathrm{H + H + gr \rightarrow H_2 + gr}$ 
        &$3.0\times 10^{-17}$ & &1, 2\\
        
        \tableline
    \end{tabular}
    \tablenotetext{1}{}
\end{table*}
